
\documentclass{screenplay}
\usepackage[utf8]{inputenc}
\title{Je suis : une rupture}
\author{Maxime Bettinelli}


\newcommand{\speak}[2]{
\begin{dialogue}{#1}
#2
\end{dialogue}
}



\begin{document}


\newcommand{\temps}{
\intslug[jour N\&B]{L'appartement, salon}
}



\maketitle


Thème du concours Nikon : "Je suis le partage"

Situation : un couple. Ils ont entre 20 et 35 ans. La scène a lieu alors qu'ils partagent, dans des cartons à leur nom, inscrit au marker, les affaires qu'il avaient jusqu'alors mises en commun.

Principe : structure en flash-back, avec chromatisme inversé par rapport à l'habitutde. Le moment (postérieur) du partage est en noir et blanc. Les flash-backs sont filmés en couleur.


\intslug[jour N\&B]{L'appartement, pièces successives}
Un appartement assez grand, élégant.  Avant d'arriver au salon, à une séries de plans sur différentes pièces de l'appartement.

Un dialogue en voix-off:

\speak{Elle}{
On est d'accord, c'est mieux comme ça.
}
\speak{Lui}{
Oui, c'est mieux.
}
\speak{Elle}{
On va pas se rendre malheureux, c'est trop idiot.
}
\speak{Lui}{
Oui, c'est mieux.
}

\textit{Les plans successifs arrivent à:}

// Cloche //
\temps
Plan cadre : ils sont tous les deux occupés à trier des choses. Au premier plan deux cartons qui portent leurs noms au marker

\speak{Elle}{
Le cendrier ? (\textit{elle tient un cendrier}) Ca doit être à moi ça.
}


\intslug[soir FB]{Un restaurant animé}
Après un dîner dans un restaurant, des amis discutent, le couple est là. 
Travelling sur les personnes : elle fume, lui non


\speak{Lui (off)}{
Non, c'est le mien. Au début, c'était moi qui fumais.
}

retour du travelling, les personnes ont potentiellement changé de vêtements, d'apparence
Il fume, elle non


\speak{Lui (off)}{
Ca fait rien, tu peux le garder
}


// Cloche //
\temps
Elle tient un vinyl

\speak{Elle}{
C'est à toi ce vinyl non ?
}
\extslug[jour FB]{Un jardin}
Ils sont assis tous les deux, elle lui offre un cadeau

\speak{Elle}{
Joyeux anniversaire!
}

Le cadeau : un vinyl de (??)

\speak{Lui}{
Merci ! Mais comment je suis censé le lire?
}

Elle lui montre une partie de la couverture : (travelling très gros plan sur les lettres du titre) une carte micro SD est collée à un endroit de la pochette

// Cloche //
\temps
Il tient une BD

\speak{Lui}{
Ce truc-là par contre c'est à toi
}

\intslug[Jour FB]{Appartement}
Elle lui tend une BD


\speak{Elle}{
Tiens, regarde, j'ai acheté ça pour toi, c'est super bien
}

\speak{Lui}{
C'est quoi ?
}




\speak{Elle}{
C'est une BD de **(gotlib??), j'adore! Attends, regarde là c'est trop marrant
}

4 plans succesifs d'elle en train de lire la BD et de rire

// Cloche //
\temps
Il sort une robe d'un placard

\speak{Lui}{
Oh, regarde, ta jolie robe
}
\speak{Elle (hors champ)}{
Connard
}

\intslug[jour FB]{Appartement}
Elle porte une robe ridicule, et il est en train de se moquer d'elle
\speak{Lui}{
.. vraiment super. T'as trouvé ça chez Monoprix non ?
}
Elle s'agace progressivement, finit par se vexer:
\speak{Elle}{
Ah ben tu dois regretter Emilie, hein
}
\speak{Lui}{
Oui, ça, c'était une salope, mais t'as raison de dire qu'elle aurait jamais porté une horreur comme ça!
}

// Cloche //
\temps
court silence. Plan sur elle qui lit un papier et sourit moqueusement progressivement

\speak{Elle (sarcastique)}{
Tiens, c'est ton poème ...
}
Il soupire

\intslug[jour FB]{Appartement}
Elle rit avec un papier à la main, qu'il essaie de lui enlever

\speak{Elle}{
... "sur le fermoir de tes lèvres" (rit) I didn't realise you wrote such bloody awful poetry!
}
Il finit par lui arracher des mains, elle continue à rire seule face à la caméra

// Cloche //
\temps

Cinq ou dix secondes pendant lesquelles ils rangent telle ou telle chose.
Elle prend un sac, regarde ce qu'il y a dedans, et finit par le déposer dans son carton

\speak{Lui}{
Tiens, c'est ce sac que tu avais quand je suis tombé amoureux de toi
}
\speak{Elle}{
Ca m'étonnerait, on étant ensemble depuis au moins un an quand je l'ai acheté
}
\speak{Lui}{
Oui, je sais. C'était un jour où tu rentrais du boulot. Je suis monté par hasard dans le même métro que toi, mais tu ne m'as pas vu. Je savais que j'aurais pu aller te voir et t'embrasser, mais, c'est étrange, tu avais l'air tellement distante que j'ai eu l'impression que tu m'aurais rejetté, comme si j'était un de ces types qui draguent dans le métro. Pendant un instant tu es devenue une jolie fille dans le métro à qui on n'ose pas adresser la parole. C'est idiot, mais j'ai cru t'avoir perdue.
Bien sûr, quand je suis rentré, tu n'avais plus du tout le même air froid. C'est en t'embrassant ce soir-là que je suis vraiment tombé amoureux
}

// Cloche //
\temps
Elle récupère un papier dans le carton de l'autre
\speak{Elle}{
Merde, c'est quoi ça ?
}

\intslug[jour FB]{Une librairie}
Il se dirige vers la caisse pour un renseignement
\speak{Le vendeur}{
Bonjour Monsieur! Je peux vous aider ?
}
\speak{Lui}{
Oui, je voudrais "L'insoutenable légerté de l'être" en folio, s'il vous plaît.
}
Tête étonnée du vendeur
\speak{Le vendeur}{
Mais enfin, Monsieur, vous n'allez tout de même pas...
}
\speak{Lui}{
Non, non, c'est juste pour la photo de la fille en couverture
}
\speak{Le vendeur}{
Ah, bon..
}
Rassuré, le vendeur prend un exemplaire, déchire la couverture, jette le livre et lui tend la couverture arrachée. Gros plan sur la couverture. La photo de la fille en N\&B

\textit{fondu vers :}

\intslug[jour N\&B]{Appartement}

Gros plan sur son visage en pleurs.

Puis plan sur lui qui se mord les lèvres. Leurs voix repètent, en off :


\speak{Elle}{
On est d'accord, c'est mieux comme ça.
}
\speak{Lui}{
Oui, c'est mieux.
}
\speak{Elle}{
On va pas se rendre malheureux, c'est trop idiot.
}
\speak{Lui}{
Oui, c'est mieux.
}




\end{document}
